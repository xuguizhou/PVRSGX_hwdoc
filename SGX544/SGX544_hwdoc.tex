\NeedsTeXFormat{LaTeX2e}

\documentclass[a4paper,12pt,
headsepline,        % Linie Kopfzeile / Text
oneside,            % einseitig
%pointlessnumbers,   % kein Punkt nach letzter Gliederungsziffer
BCOR15mm          % mehr li. Rand zum Binden der Arbeit
]{scrbook}


\KOMAoptions{DIV=last}
\usepackage[a4paper, left=1cm, right=1cm, top=2cm, bottom=3cm]{geometry}
\pagestyle{headings}
\usepackage{blindtext}
\usepackage{float}
\usepackage{subcaption}
\usepackage{listings}
\usepackage{booktabs}
\usepackage[english]{babel}
\usepackage[utf8]{inputenc}
\usepackage[T1]{fontenc}

\usepackage{textcomp}
\usepackage{gensymb}

% Helvetica als Standard-Dokumentschrift
\usepackage[scaled]{helvet}
\renewcommand{\familydefault}{\sfdefault}
\usepackage[export]{adjustbox}
\usepackage{graphicx}
\usepackage{epstopdf}
\usepackage{array}
\usepackage{tabularx}
\usepackage{longtable}
\usepackage{dcolumn}
\usepackage{url}              % \url{http://...} in Schreibmaschinenschrift
\usepackage{color}            % zum Setzen farbigen Textes
\usepackage{amssymb, amsmath} % Pakete für Mathe-Umgebungen und -Symbole
\usepackage{setspace} % Paket für div. Abstände, z.B. ZA

% hier Namen etc. einsetzen
\newcommand{\SGXtype}{SGX544}
\newcommand{\fullname}{Philipp Rossak}
\newcommand{\email}{embed3d@gmail.com}

\newcommand{\titel}{Reverse-Engineered Datasheet PowerVR \SGXtype}
\newcommand{\titeltext}{Reverse-Engineered Datasheet\\* PowerVR \SGXtype}
\newcommand{\jahr}{2017}
% Table Defintions
\newcommand{\regdscBit}{0.1}
\newcommand{\regdscRW}{0.1}
\newcommand{\regdscHex}{0.15}
\newcommand{\regdscDesc}{0.55}

%color in tables
\usepackage{colortbl}
\definecolor{Gray}{rgb}{0.80784, 0.86667, 0.90196} %dunkelblau
\definecolor{Lightgray}{rgb}{0.9176, 0.95, 0.95686} %hellblau
\definecolor{Akzent}{rgb}{0.6627, 0.63529, 0.55294} %akzentfarbe
\setlength{\arrayrulewidth}{0.1pt}


\setlength{\parindent}{0pt}
\setlength{\parskip}{1.4ex plus 0.35ex minus 0.3ex}

% Tiefe, bis zu der Überschriften in das Inhaltsverzeichnis kommen
\setcounter{tocdepth}{3} % ist Standard

\pdfinfo{
	/Author (\fullname)
	/Title (\titel)
	/Producer     (pdfeTex 3.14159-1.30.6-2.2)
	/Keywords ()
}

\usepackage{hyperref}
\hypersetup{
	pdftitle=\titel,
	pdfauthor=\fullname,
	pdfsubject={Labor Bericht},
	pdfproducer={pdfeTex 3.14159-1.30.6-2.2},
	colorlinks=false,
	pdfborder=0 0 0	% keine Box um die Links!
}
\newcolumntype{C}[1]{>{\centering\arraybackslash}m{#1}}
\newcolumntype{M}[1]{>{\arraybackslash}m{#1}}
%Trennungsregeln
\hyphenation{Sil-ben-trenn-ung}

\usepackage{color} %red, green, blue, yellow, cyan, magenta, black, white
\definecolor{mygreen}{RGB}{28,172,0} % color values Red, Green, Blue
\definecolor{mylilas}{RGB}{170,55,241}

\begin{document}

	\frontmatter
	
	% Titelseite
	\thispagestyle{empty}
	\begin{addmargin*}[4mm]{-10mm}
		
		\includegraphics[height=5cm]{images/osi_standard_logo}
		\hspace{60mm}
		\includegraphics[height=5cm]{images/gerwinski-gnu-head}
		
		\vspace{60mm}
		{\footnotesize
			
			\parbox{140mm}{\bfseries \LARGE \titeltext}\\[2.5em]
			
			{\footnotesize \bfseries Executed by:}\\
			{\footnotesize \fullname\\\email}\\[2em]
			{\footnotesize\jahr}
		}
	\end{addmargin*}
	
	% ab hier Zeilenabstand etwas größer
	\setstretch{1.5}
	
	\tableofcontents
	
	\mainmatter
	\thispagestyle{empty}
	\newpage
	\input{chapters/introduction}
	\chapter{Register List}\label{chp:reglist}

These information come from Published code from Img Tech and is under the GPL v2 Licencse\\
According to the Datasheet of the A83T the Registers have a total size of 64k (Adressrange: 0x0000 - 0xFFFF)

\begin{longtable}[c]{|m{0.4\textwidth}|C{0.10\textwidth}|m{0.42\textwidth}|} \hline
	Register Name & Offset & Description \\ \hline
	CLKGATECTL & 0x0000 & Clock Gate Control Register \\ \hline
	CLKGATECTL2 & 0x0004 & Clock Gate Control Register2 \\ \hline
	CLKGATESTATUS & 0x0008 & Clock Gate Status Register \\ \hline
	CLKGATECTLOVR & 0x000C & Clock Gate -?- Register \\ \hline
	POWER  & 0x001C & Power Register \\ \hline
	CORE\_ID & 0x0020 & Core ID Register \\ \hline
	CORE\_REVISION & 0x0024 & Core Revision Register\\ \hline
	DESIGNER\_REV\_FIELD1 & 0x0028 & Designer Revision Field 1 Register \\ \hline
	DESIGNER\_REV\_FIELD2 & 0x002C & Designer Revision Field 2 Register \\ \hline
	SOFT\_RESET & 0x0080 & Soft Reset Register \\ \hline
	EVENT\_HOST\_ENABLE2 & 0x0110 & Event Host Enable Register 2 \\ \hline
	EVENT\_HOST\_CLEAR2 & 0x0114 & Event Host Clear Register 2 \\ \hline
	EVENT\_STATUS2 & 0x0118 & Event Status Register 2 \\ \hline
	EVENT\_STATUS1 & 0x012C & Event Status Register 1 \\ \hline
	EVENT\_HOST\_ENABLE1 & 0x0130 & Event Host Enable Register 1 \\ \hline
	HOST\_CLEAR1 & 0x0134  & Event Host Clear Register 1 \\ \hline
	TIMER & 0x0144 & Timer Register \\ \hline
	EVENT\_KICK1 & 0x0AB0 & Event Kick 1 Register \\ \hline
	EVENT\_KICK2 & 0x0AC0 & Event Kick 2 Register\\ \hline
	EVENT\_KICKER & 0x0AC4 & Event Kicker Register \\ \hline
	EVENT\_KICK0 & 0x0AC8 & Event Kick 0 Register \\ \hline
	EVENT\_TIMER & 0x0ACC & Event Timer Register \\ \hline
	PDS\_INV0 & 0x0AD0 & -?- \\ \hline
	PDS\_INV1 & 0x0AD4 & -?- \\ \hline
	EVENT\_KICK3 & 0x0AD8 & Event Kick 3 Register \\ \hline
	PDS\_INV3 & 0x0ADC & -?- \\ \hline
	PDS\_INV\_CSC & 0x0AE0 & -?- \\ \hline
	BIF\_CTRL & 0x0C00 & BIF Control Register \\ \hline
	BIF\_INT\_STAT  & 0x0C04 & BIF INT Status Register \\ \hline
	BIF\_FAULT  & 0x0C08 & BIF Fault Register \\ \hline
	BIF\_TILE0 & 0x0C0C & BIF Tile 0 Register \\ \hline
	BIF\_TILE1 & 0x0C10 & BIF Tile 1 Register \\ \hline
	BIF\_TILE2 & 0x0C14 & BIF Tile 2 Register \\ \hline
	BIF\_TILE3 & 0x0C18 & BIF Tile 3 Register \\ \hline
	BIF\_TILE4 & 0x0C1C & BIF Tile 4 Register \\ \hline
	BIF\_TILE5 & 0x0C20 & BIF Tile 5 Register \\ \hline
	BIF\_TILE6 & 0x0C24 & BIF Tile 6 Register \\ \hline
	BIF\_TILE7 & 0x0C28 & BIF Tile 7 Register \\ \hline
	BIF\_TILE8 & 0x0C2C & BIF Tile 8 Register \\ \hline
	BIF\_TILE9 & 0x0C30 & BIF Tile 9 Register \\ \hline
	BIF\_CTRL\_INVAL & 0x0C34 & BIF Control INVAL Register \\ \hline
	BIF\_DIR\_LIST\_BASE 1 & 0x0C38 & -?- \\ \hline
	BIF\_DIR\_LIST\_BASE 2 & 0x0C3C & -?- \\ \hline
	BIF\_DIR\_LIST\_BASE 3 & 0x0C40 & -?- \\ \hline
	BIF\_DIR\_LIST\_BASE 4 & 0x0C44 & -?- \\ \hline
	BIF\_DIR\_LIST\_BASE 5 & 0x0C48 & -?- \\ \hline
	BIF\_DIR\_LIST\_BASE 6 & 0x0C4C & -?- \\ \hline
	BIF\_DIR\_LIST\_BASE 7 & 0x0C50 & -?- \\ \hline
	BIF\_BANK\_SET & 0x0C74 & BIF Bank Set Register \\ \hline
	BIF\_BANK0 & 0x0C78 & BIF Bank 0 Register \\ \hline
	BIF\_BANK1 & 0x0C7C & BIF Bank 1 Register \\ \hline
	BIF\_DIR\_LIST\_BASE0 & 0x0C84 & -?- \\ \hline
	BIF\_TA\_REQ\_BASE & 0x0C90 & -?- \\ \hline
	BIF\_MEM\_REQ\_STAT & 0x0CA8 & -?- \\ \hline
	BIF\_3D\_REQ\_BASE & 0x0CAC & -?- \\ \hline
	BIF\_ZLS\_REQ\_BASE & 0x0CB0 & -?- \\ \hline
	BIF\_BANK\_STATUS & 0x0CB4 & -?- \\ \hline
	BIF\_MMU\_CTRL & 0x0CD0 & -?- \\ \hline
	2D\_BLIT\_STATUS & 0x0E04 & BLIT Status Register \\ \hline
	2D\_VIRTUAL\_FIFO\_0 & 0x0E10 & -?- \\ \hline
	2D\_VIRTUAL\_FIFO\_1 & 0x0E14 & -?- \\ \hline
	BREAKPOINT0\_START & 0x0F44 & -?- \\ \hline
	BREAKPOINT0\_END & 0x0F48 & -?- \\ \hline
	BREAKPOINT0 & 0x0F4C & -?- \\ \hline
	BREAKPOINT1\_START & 0x0F50 & -?- \\ \hline
	BREAKPOINT1\_END & 0x0F54 & -?- \\ \hline
	BREAKPOINT1 & 0x0F58 & -?- \\ \hline
	BREAKPOINT2\_START & 0x0F5C & -?- \\ \hline
	BREAKPOINT2\_END & 0x0F60 & -?- \\ \hline
	BREAKPOINT2 & 0x0F64 & -?- \\ \hline
	BREAKPOINT3\_START & 0x0F68 & -?- \\ \hline
	BREAKPOINT3\_END & 0x0F6C & -?- \\ \hline
	BREAKPOINT3 & 0x0F70 & -?- \\ \hline
	BREAKPOINT\_READ & 0x0F74 & -?- \\ \hline
	PARTITION\_BREAKPOINT\_TRAP & 0x0F78 & -?- \\ \hline
	PARTITION\_BREAKPOINT & 0x0F7C & -?- \\ \hline
	PARTITION\_BREAKPOINT\_TRAP\_INFO0 & 0x0F80 & -?- \\ \hline
	PARTITION\_BREAKPOINT\_TRAP\_INFO1 & 0x0F84 & -?- \\ \hline
	USE\_CODE\_BASE\_0 & 0x0A0C & Use Code Base Register 0   \\ \hline
	USE\_CODE\_BASE\_1 & 0x0A10 & Use Code Base Register 1   \\ \hline
	USE\_CODE\_BASE\_2 & 0x0A14 & Use Code Base Register 2   \\ \hline
	USE\_CODE\_BASE\_3 & 0x0A18 & Use Code Base Register 3   \\ \hline
	USE\_CODE\_BASE\_4 & 0x0A1C & Use Code Base Register 4   \\ \hline
	USE\_CODE\_BASE\_5 & 0x0A20 & Use Code Base Register 5   \\ \hline
	USE\_CODE\_BASE\_6 & 0x0A24 & Use Code Base Register 6   \\ \hline
	USE\_CODE\_BASE\_7 & 0x0A28 & Use Code Base Register 7   \\ \hline
	USE\_CODE\_BASE\_8 & 0x0A2C & Use Code Base Register 8   \\ \hline
	USE\_CODE\_BASE\_9 & 0x0A30 & Use Code Base Register 9   \\ \hline
	USE\_CODE\_BASE\_10 & 0x0A34 & Use Code Base Register 10 \\ \hline
	USE\_CODE\_BASE\_11 & 0x0A38 & Use Code Base Register 11 \\ \hline
	USE\_CODE\_BASE\_12 & 0x0A3C & Use Code Base Register 12 \\ \hline
	USE\_CODE\_BASE\_13 & 0x0A40 & Use Code Base Register 13 \\ \hline
	USE\_CODE\_BASE\_14 & 0x0A44 & Use Code Base Register 14 \\ \hline
	USE\_CODE\_BASE\_15 & 0x0A48 & Use Code Base Register 15 \\ \hline
	PIPE0\_BREAKPOINT\_TRAP & 0x0F88 & Pipe 0 Breakpoint Trap Register \\ \hline
	PIPE0\_BREAKPOINT  & 0x0F8C & Pipe 0 Breakpoint Register \\ \hline
	PIPE0\_BREAKPOINT\_TRAP\_INFO0 & 0x0F90 & Pipe 0 Breakpoint Trap Info Register 0 \\ \hline
	PIPE0\_BREAKPOINT\_TRAP\_INFO1 & 0x0F94 & Pipe 0 Breakpoint Trap Info Register 1\\ \hline
	PIPE1\_BREAKPOINT\_TRAP & 0x0F98 & Pipe 1 Breakpoint Trap Register \\ \hline
	PIPE1\_BREAKPOINT & 0x0F9C & Pipe 1 Breakpoint Register \\ \hline
	PIPE1\_BREAKPOINT\_TRAP\_INFO0 & 0x0FA0 & Pipe 1 Breakpoint Trap Info Register 0 \\ \hline
	PIPE1\_BREAKPOINT\_TRAP\_INFO1 & 0x0FA4 & Pipe 1 Breakpoint Trap Info Register 1 \\ \hline
	\caption{Register List \SGXtype}
	\label{tab:reglist}
\end{longtable}
\newpage
\begin{longtable}[c]{|m{0.4\textwidth}|C{0.10\textwidth}|m{0.42\textwidth}|} \hline
	Register Name & Offset & Description \\ \hline
	MASTER\_BIF\_CRTL & 0x4C00 & Master BIF Control Register \\ \hline
	MASTER\_BIF\_INIT\_STAT & 0x4C04 & Master BIF -?- Register \\ \hline
	\caption{Register List MP Master}
	\label{tab:reglist_mp_feature}
\end{longtable}

	\chapter{Register Descritption}

\begin{longtable}[c]{ |C{\regdscBit\textwidth}|C{\regdscRW\textwidth}|C{\regdscHex\textwidth}|p{\regdscDesc\textwidth}| } \hline
	\multicolumn{3}{ |c| }{Offset: } &  Registername: \textbf{REGNAME} \\ \hline
	Bit & R/W & Default/Hex & Description \\ \hline
	31:0 & R/W & 0xFFFFFFFF & NAMEREGISTER \newline \\ \hline
	\caption{Example table \SGXtype}
	\label{tab:sample}
\end{longtable}


\section{Clock Gate Control Register (Default Value: 0xXXXXXXXX)}

\begin{longtable}[c]{ |C{\regdscBit\textwidth}|C{\regdscRW\textwidth}|C{\regdscHex\textwidth}|p{\regdscDesc\textwidth}| } \hline
	\multicolumn{3}{ |c| }{Offset: 0x0000} &  Registername: \textbf{CLKGATECTL} \\ \hline
	Bit & R/W & Default/Hex & Description \\ \hline
	31:29 & / & / & / \\ \hline
	28 & R/W &  & CLKGATECTL\_SYSTEM\_CLKG \newline  1: -?-  \newline 0: -?- \\ \hline
	27:25 & / & / & / \\ \hline
	24 & R/W &  & CLKGATECTL\_AUTO\_MAN\_REG \newline \\ \hline
	23:22 & / & / & / \\ \hline
	21:20 & R/W & 0x0 & CLKGATECTL\_BIF\_CORE\_CLKG \newline 0x0: -?- \newline 0x1: -?- \newline 0x2: -?- \\ \hline
	19:18 & R/W & 0x0 & CLKGATECTL\_TA\_CLKG \newline 0x0: -?- \newline 0x1: -?- \newline 0x2: -?- \\ \hline
	17:16 & R/W & 0x0 & CLKGATECTL\_IDXFIFO\_CLKG \newline 0x0: -?- \newline 0x1: -?- \newline 0x2: -?- \\ \hline
	15:14 & R/W & 0x0 & CLKGATECTL\_PDS\_CLKG \newline 0x0: -?- \newline 0x1: -?- \newline 0x2: -?- \\ \hline
	13:12 & R/W & 0x0 & CLKGATECTL\_VDM\_CLKG \newline 0x0: -?- \newline 0x1: -?- \newline 0x2: -?- \\ \hline
	11:10& R/W & 0x0 & CLKGATECTL\_DPM\_CLKG \newline 0x0: -?- \newline 0x1: -?- \newline 0x2: -?- \\ \hline
	9:8 & R/W & 0x0 & CLKGATECTL\_MTE\_CLKG \newline 0x0: -?- \newline 0x1: -?- \newline 0x2: -?- \\ \hline
	7:6 & R/W & 0x0 & CLKGATECTL\_TE\_CLKG \newline 0x0: -?- \newline 0x1: -?- \newline 0x2: -?- \\ \hline
	5:4 & R/W & 0x0 & CLKGATECTL\_TSP\_CLKG \newline 0x0: -?- \newline 0x1: -?- \newline 0x2: -?- \\ \hline
	3:2 & R/W & 0x0 & CLKGATECTL\_ISP2\_CLKG \newline 0x0: -?- \newline 0x1: -?- \newline 0x2: -?- \\ \hline
	1:0 & R/W & 0x0 & CLKGATECTL\_ISP\_CLKG \newline 0x0: -?- \newline 0x1: -?- \newline 0x2: -?- \\ \hline
	\caption{Clock Gate Control Register}
	\label{tab:reg_clk_gate_ctl}
\end{longtable}

\section{Clock Gate Control Register 2 (Default Value: 0xXXXXXXXX)}
\begin{longtable}[c]{ |C{\regdscBit\textwidth}|C{\regdscRW\textwidth}|C{\regdscHex\textwidth}|p{\regdscDesc\textwidth}| } \hline
	\multicolumn{3}{ |c| }{Offset: 0x0004} &  Registername: \textbf{CLKGATECTL2} \\ \hline
	Bit & R/W & Default/Hex & Description \\ \hline
	31:0 & R/W &  & NAME\_REGISTER \newline \\ \hline
	\caption{Clock Gate Control Register 2}
	\label{tab:reg_clk_gate_ctl2}
\end{longtable}
\section{Clock Gate Status Register (Default Value: 0xXXXXXXXX)}

\begin{longtable}[c]{ |C{\regdscBit\textwidth}|C{\regdscRW\textwidth}|C{\regdscHex\textwidth}|p{\regdscDesc\textwidth}| } \hline
	\multicolumn{3}{ |c| }{Offset: 0x0008} &  Registername: \textbf{CLKGATESTATUS} \\ \hline
	Bit & R/W & Default/Hex & Description \\ \hline
	31:0 & R/W &  & NAMEREGISTER \newline \\ \hline
	\caption{Clock Gate Status Register}
	\label{tab:reg_clk_gate_status}
\end{longtable}
\section{Clock Gate -?- Register (Default Value: 0xXXXXXXXX)}

\begin{longtable}[c]{ |C{\regdscBit\textwidth}|C{\regdscRW\textwidth}|C{\regdscHex\textwidth}|p{\regdscDesc\textwidth}| } \hline
	\multicolumn{3}{ |c| }{Offset: 0x000C} &  Registername: \textbf{CLKGATECTLOVR} \\ \hline
	Bit & R/W & Default/Hex & Description \\ \hline
	31:0 & R/W &  & NAMEREGISTER \newline \\ \hline
	\caption{Clock Gate -?- Register}
	\label{tab:reg_gate_ctl_ovr}
\end{longtable}
\section{Power Register (Default Value: 0xXXXXXXXX)}

\begin{longtable}[c]{ |C{\regdscBit\textwidth}|C{\regdscRW\textwidth}|C{\regdscHex\textwidth}|p{\regdscDesc\textwidth}| } \hline
	\multicolumn{3}{ |c| }{Offset: 0x001C} &  Registername: \textbf{POWER} \\ \hline
	Bit & R/W & Default/Hex & Description \\ \hline
	31:1 & / & / & / \\ \hline
	0 & R/W & 0x0 & POWER\_PIPE\_DISABLE \newline Disable -?- \newline 1: Disable ? \newline 0: Enable ? \\ \hline
	\caption{Power Register}
	\label{tab:reg_power}
\end{longtable}
\section{Core ID Register (Default Value: 0xXXXXXXXX)}

\begin{longtable}[c]{ |C{\regdscBit\textwidth}|C{\regdscRW\textwidth}|C{\regdscHex\textwidth}|p{\regdscDesc\textwidth}| } \hline
	\multicolumn{3}{ |c| }{Offset: 0x0020} &  Registername: \textbf{CORE\_ID} \\ \hline
	Bit & R/W & Default/Hex & Description \\ \hline
	31:16 & R & / & CORE\_ID\_ID  \newline \\ \hline
	15:12 & R & /& CORE\_ID\_CONFIG\_SLC \newline \\ \hline
	11:81& R & / & SGX531 CORE\_ID\_CONFIG\_CORES  \newline \\ \hline
	7:2 & R & / & CORE\_ID\_CONFIG \newline \\ \hline
	1 & R & / & CORE\_ID\_CONFIG\_BASE \newline \\ \hline
	0 & R & / & CORE\_ID\_CONFIG\_MULTI \newline  \\ \hline
	\caption{Core ID Register}
	\label{tab:reg_core_id}
\end{longtable}


\section{Core Revision Register (Default Value: 0xXXXXXXXX)}

\begin{longtable}[c]{ |C{\regdscBit\textwidth}|C{\regdscRW\textwidth}|C{\regdscHex\textwidth}|p{\regdscDesc\textwidth}| } \hline
	\multicolumn{3}{ |c| }{Offset: 0x0024} &  Registername: \textbf{CORE\_REVISION} \\ \hline
	Bit & R/W & Default/Hex & Description \\ \hline
	31:24 & R & / & CORE\_REVISION\_DESIGNER \newline \\ \hline
	23:16 & R & / & CORE\_REVISION\_MAJOR \newline \\ \hline
	15:8 & R & / & CORE\_REVISION\_MINOR \newline \\ \hline
	7:0 & R & / & CORE\_REVISION\_MAINTENANCE \newline  \\ \hline
	\caption{Core Revision Register}
	\label{tab:reg_core_rev}
\end{longtable}

Revison : Major.Minor.Maintenance\\
A83T Revision: 1.1.5
\section{Core Revision Field 1 Register (Default Value: 0xXXXXXXXX)}

\begin{longtable}[c]{ |C{\regdscBit\textwidth}|C{\regdscRW\textwidth}|C{\regdscHex\textwidth}|p{\regdscDesc\textwidth}| } \hline
	\multicolumn{3}{ |c| }{Offset: 0x0028} &  Registername: \textbf{CORE\_REV\_FIELD1} \\ \hline
	Bit & R/W & Default/Hex & Description \\ \hline
	31:0 & R & / & DESIGNER\_REV\_FIELD1\_DESIGNER\_REV\_FIELD1 \newline ? \\ \hline
	\caption{Core Revision Field 1 Register}
	\label{tab:reg_core_rev_f1}
\end{longtable}
\section{Core Revision Field 2 Register (Default Value: 0xXXXXXXXX)}

\begin{longtable}[c]{ |C{\regdscBit\textwidth}|C{\regdscRW\textwidth}|C{\regdscHex\textwidth}|p{\regdscDesc\textwidth}| } \hline
	\multicolumn{3}{ |c| }{Offset: 0x002C} &  Registername: \textbf{CORE\_REV\_FIELD2} \\ \hline
	Bit & R/W & Default/Hex & Description \\ \hline
	31:0 & R & / & DESIGNER\_REV\_FIELD2\_DESIGNER\_REV\_FIELD2 \newline ?\\ \hline
	\caption{Core Revision Field 2 Register}
	\label{tab:reg_core_rev_f2}
\end{longtable}
	

	\listoffigures
    \listoftables



\end{document}

	
